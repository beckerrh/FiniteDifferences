\documentclass[12pt, english]{article}
%----------------------------------------
\usepackage[a4paper, top=1cm, bottom=2cm]{geometry}
\usepackage{graphicx, xcolor}
\usepackage{amssymb,amsmath}
\usepackage{url}
\definecolor{mygray}{rgb}{0.9,0.9,0.9}
\definecolor{mygreen}{rgb}{0,0.6,0}

\usepackage{listings}
\lstset{ 
  backgroundcolor=\color{mygray},
  basicstyle=\ttfamily\footnotesize,
  breaklines=true,
  commentstyle=\color{mygreen},
  deletekeywords={...},
  escapeinside={\%*}{*)},
  frame=single,
  showstringspaces=false,
  language=Python
}
\usepackage[many]{tcolorbox}
\tcbuselibrary{breakable}

%---------------------------------------------------------
%\newcommand{\comment}[1]{\textcolor{blue}{\ #1}}
\newcommand{\disp}[1]{$\displaystyle #1$}
%---------------------------------------------------------
%
%
\newcommand{\dirder}[2]{\frac{\partial #1}{\partial #2} }
\newcommand{\betagrad}[1]{\dirder{#1}{\beta} }
%----------------------------------------
% FEM
\newcommand{\ind}{{\tt ind}}
\newcommand{\h}{\mathsf{h}} 
\newcommand{\Cspace}{\mathcal{C}}
\newcommand{\Dspace}{\mathcal{D}}
\newcommand{\Pspace}{\mathcal{P}}
\newcommand{\CR}{\mathcal{C\!R}}
\newcommand{\RT}{\mathcal{R\!T}}
\newcommand{\Pkhom}[1]{P_{\rm hom}^{#1}}
\newcommand{\Phom}{\Pkhom{k}}
%----------------------------------------
% numbers
\newcommand{\R}{\mathbb R}
\newcommand{\N}{\mathbb N}
\newcommand{\C}{\mathbb C}
\newcommand{\Z}{\mathbb Z}
%----------------------------------------
% sets and functions
\newcommand{\Set}[1]{\left\{#1\right\}} 
\newcommand{\SetDef}[2]{\left\{#1\;\middle|\;#2\right\}} 
\newcommand{\vect}[1]{\operatorname{Vect}\Set{#1}}
\newcommand{\supp}[1]{\operatorname{supp}(#1)}
\newcommand{\abs}[1]{\left|#1\right|} 
\newcommand{\norm}[1]{\left\|#1\right\|}
\newcommand{\tnorm}[1]{\left\||#1|\right\|}
\newcommand{\eps}{\varepsilon}
\newcommand{\scp}[2]{\left\langle#1,#2\right\rangle}
\newcommand{\sgn}[1]{\operatorname{sgn}(#1)}
\newcommand{\conv}[1]{\operatorname{conv}\Set{#1}}
\newcommand{\convdef}[2]{\operatorname{conv}\SetDef{#1}{#2}}
\newcommand{\Rest}[2]{{#1}{|_{#2}}}
\makeatletter
\newcommand{\opnorm}{\@ifstar\@opnorms\@opnorm}
\newcommand{\@opnorms}[1]{%
  \left|\mkern-1.5mu\left|\mkern-1.5mu\left|
   #1
  \right|\mkern-1.5mu\right|\mkern-1.5mu\right|
}
\newcommand{\@opnorm}[2][]{%
  \mathopen{#1|\mkern-1.5mu#1|\mkern-1.5mu#1|}
  #2
  \mathclose{#1|\mkern-1.5mu#1|\mkern-1.5mu#1|}
}
\makeatother
%----------------------------------------
% linear algebra
\newcommand{\Ref}[1]{#1_{\rm ref}} 
\newcommand{\transpose}[1]{{#1}^{\mathsf{T}}} 
\newcommand{\transposeInv}[1]{{#1}^{\mathsf{-T}}} 
\newcommand{\trace}{\operatorname{tr}} 
\newcommand{\adj}{\operatorname{adj}} 
\newcommand{\diag}{\operatorname{diag}}
\newcommand{\sym}[1]{#1_{\mathsf{s}}} 
\newcommand{\asym}[1]{#1_{\mathsf{a}}} 
\newcommand{\id}{\operatorname{id}} 
%----------------------------------------
% diff
\renewcommand{\div}{\operatorname{div}}
\newcommand{\grad}{\operatorname{grad}}
\newcommand{\gradS}{\sym{\operatorname{grad}}}
\newcommand{\rot}{\operatorname{rot}}
\newcommand{\dd}[2]{\frac{\partial #1}{\partial #2}}
\newcommand{\dn}[1]{\dd{#1}{n}}
\newcommand{\dt}[1]{\dd{#1}{t}}
%----------------------------------------
% meshes
\newcommand{\allmeshes}{\mathcal H(\Omega)}
\newcommand{\Cells}{\mathcal K}
\newcommand{\Sides}{\mathcal S}
\newcommand{\Faces}{\mathcal F}
\newcommand{\Nodes}{\mathcal N}
\newcommand{\NodesInt}{{\mathcal N}^{int}}
\newcommand{\level}{\operatorname{lev}}
\newcommand{\SidesInt}{\mathcal S^{\rm int}}
\newcommand{\SidesBound}{\mathcal S^{\partial}}
\newcommand{\SidesBdry}{\mathcal S^{\partial}}
\newcommand{\FacesInt}{\mathcal F^{\rm int}}
\newcommand{\FacesBdry}{\mathcal F^{\partial}}
%
\newcommand{\In}[1]{#1^{\rm in}}
\newcommand{\Ex}[1]{#1^{\rm ex}}
\newcommand{\InDe}[2]{#1^{{\rm in}_{#2}}}
\newcommand{\ExDe}[2]{#1^{{\rm ex}_{#2}}}
%
\newcommand{\Kin}{K^{\rm{\footnotesize{in}}}}
\newcommand{\Kex}{K^{\rm{\footnotesize{ex}}}}
\newcommand{\inS}[1]{{#1}^{\rm{\footnotesize{in}}}_{S}}
\newcommand{\exS}[1]{{#1}^{\rm{\footnotesize{ex}}}_{S}}
\newcommand{\inSs}[1]{{#1}^{\rm{\footnotesize{in}}}}
\newcommand{\exSs}[1]{{#1}^{\rm{\footnotesize{ex}}}}
\newcommand{\meanS}[1]{\{#1\}_{S}}
\newcommand{\jumpS}[1]{\left[#1\right]_{S}}
\newcommand{\jump}[1]{\left[#1\right]}
\newcommand{\mean}[1]{\left\{#1\right\}}
\newcommand{\intS}{\int_{\Sides_{h}}}
\newcommand{\intSInt}{\int_{\SidesInt_{h}}}
\newcommand{\intSBound}{\int_{\SidesBound_{h}}}
\newcommand{\intK}{\int_{\Cells_{h}}}
%
%
\newcommand{\meshh}{\rm{h}}
%----------------------------------------
% domain
\newcommand{\GammaD}{\Gamma_{\rm D}}
\newcommand{\GammaN}{\Gamma_{\rm N}}
\newcommand{\udir}{u^{\rm D}}
\newcommand{\pdir}{p^{\rm D}}
\newcommand{\uD}{u^{\rm D}}
\newcommand{\pD}{p^{\rm D}}
\newcommand{\kinv}{k^{\rm inv}}
%
%----------------------------------------
% spaces
%\newcommand{\Hdiv}[1]{H\left({\rm div},{#1}\right)}
%
%----------------------------------------
% fct-spaces
\newcommand{\Hdiv}[1]{H({\rm div},#1)}
\newcommand{\HdivO}{\Hdiv{\Omega}}
\newcommand{\prol}{\operatorname{Pr}}
%


%----------------------------------------
\title{Finite difference}
\author{Roland Becker}

%==========================================
\begin{document}
%==========================================
\maketitle
%
%
%==========================================
\section{Grid}\label{sec:}
%==========================================
%
%
\begin{equation}\label{eq:}
x_{n+1} = x_n + \omega w_n,\quad  w_n = B r_n, \quad r_n := b - A x_n
\end{equation}
%
If $A$ is SPD, we can either minimize with respect to the norm $\norm{A\cdot}$ or $\norm{A^{\frac12}\cdot}$
%
\begin{align*}
\omega^{(1)} = \frac{\transpose{r_n}{Aw_n}}{\transpose{(Aw_n)}{Aw_n}},\quad
\omega^{(2)} = \frac{\transpose{r_n}{w_n}}{\transpose{(Aw_n)}{w_n}}
\end{align*}
%
If $A$ is not symmetric, $\omega^{(2)}$ is still be well defined, if $A$ is elliptic ($\transpose{\xi}A\xi\ge \alpha\transpose{\xi}\xi$) and corresponds to minimization over the space $\vect{x_n, w_n}$.

Now let us consider a Gauss-Seidel-type iteration with $A=L+U$ and
%
\begin{align*}
x_{n+1} = (1-\omega)x_n + \omega L^{-1}(b- U x^n) = x_n +  \omega L^{-1}r^n.
\end{align*}
%


%
%==========================================
\section{Mixed FEM}\label{sec:}
%==========================================
%

%
%==========================================
\section{Grid}\label{sec:}
%==========================================
%
We suppose the following numbering
%
\begin{equation}\label{eq:}
ii = \sum_{j=0}^{d-1}\left(\prod_{k=j+1}^{d-1}n_k\right) i_j,\quad i=[i_0,\ldots,i_{d-1}]
\end{equation}
%
%
%
%==========================================
\section{Mixed FEM}\label{sec:}
%==========================================
%
The mixed formulation on a $d$-dimensional brick leads to
%
\begin{equation}\label{eq:structfemsys}
\begin{bmatrix}
A_1 && &  & B_1 \\
 & A_2 &&  & B_2 \\
&&\ddots&&\vdots  \\
& &  &  A_d& B_d \\
C_1 & C_2 &\cdots & C_d & D
\end{bmatrix}
\begin{bmatrix}
  u_1 \\ u_2 \\ \vdots \\ u_d \\ p
\end{bmatrix}
=
\begin{bmatrix}
  g_1 \\ g_2 \\ \vdots \\ g_d \\ f
\end{bmatrix}
\end{equation}
which leads to the pressure equation
\begin{equation}\label{eq:structfemsys}
  S p = f - \sum_{i=1}^d C_i A_i^{-1} g_i,\quad S := D - \sum_{i=1}^d C_i A_i^{-1}B_i
\end{equation}
  which allows to recover the fluxes by
\begin{equation*}
  A_i u_i = g_i  - B_i p.
\end{equation*}

%
%~~~~~~~~~~~~~~~~~~~~~~~~~~~~~
\subsubsection{Elimination in $d=1$}
%~~~~~~~~~~~~~~~~~~~~~~~~~~~~~
%
We have the following equations
%
\begin{equation}\label{eq:}
%
\left\{
\begin{aligned}
a_{i,i-1} u_{i-1} + a_{i,i} u_{i} + a_{i,i+1} u_{i+1} + b_{i,i-\frac12} p_{i-\frac12}   + b_{i,i+\frac12} p_{i+\frac12} =& g_i\\
c_{i-\frac12, i-1} u_{i-1} + c_{i-\frac12, i} u_{i} =& f_{i-\frac12}\\
c_{i+\frac12, i} u_{i} + c_{i+\frac12, i+1} u_{i+1} =& f_{i+\frac12}
\end{aligned}
\right.
%
\end{equation}
%
We can use the last two equations to eliminate $u_{i\pm1}$, thus
%
\begin{equation}\label{eq:}
x_i u_i =  g_i - \frac{a_{i,i-1}}{c_{i-\frac12, i-1}}f_{i-\frac12}- \frac{a_{i,i+1}}{c_{i+\frac12, i+1}}f_{i+\frac12}
- b_{i,i-\frac12} p_{i-\frac12}   - b_{i,i+\frac12} p_{i+\frac12},\quad
x_i = a_{i,i} - \frac{a_{i,i-1}c_{i-\frac12, i}}{c_{i-\frac12, i-1}}- \frac{a_{i,i+1}c_{i+\frac12, i}}{c_{i+\frac12, i+1}}
\end{equation}
%
For a boundary node we have, say the left, $i=0$, we have
\begin{equation}\label{eq:}
%
\left\{
\begin{aligned}
a_{0,0} u_{0} + a_{0,1} u_{1} + b_{0,\frac12} p_{\frac12} =& g_0\\
c_{\frac12, 0} u_{0} + c_{\frac12, 1} u_{1} =& f_{\frac12}
\end{aligned}
\right.
%
\end{equation}
%
Using the last  equation to eliminate $u_{1}$, thus
%
\begin{equation}\label{eq:}
x_0 u_0 =  g_0 - \frac{a_{0,1}}{c_{\frac12, 1}}f_{\frac12}
  - b_{0,\frac12} p_{\frac12},\quad
x_0 = a_{0,0} -  \frac{a_{0,1}c_{\frac12, 0}}{c_{\frac12, 1}}
\end{equation}
%
This gives the following finite difference stencil on the boundary
%
%
\begin{align*}
\frac{c_{\frac12, 0}}{x_{0}} \left(g_0 - \frac{a_{0,1}}{c_{\frac12, 1}}f_{\frac12}
  - b_{0,\frac12} p_{\frac12}\right) 
  + \frac{c_{\frac12, 1}}{x_1} \left(
  g_1 - \frac{a_{1,0}}{c_{\frac12, 0}}f_{\frac12}- \frac{a_{1,2}}{c_{\frac32, 2}}f_{\frac32}
- b_{1,\frac12} p_{\frac12}   - b_{1,\frac32} p_{\frac32}
\right) =& f_{\frac12}
\end{align*}
%
and on the interior
%
\begin{align*}
&\frac{c_{i+\frac12, i}}{x_i} \left( 
g_i - \frac{a_{i,i-1}}{c_{i-\frac12, i-1}}f_{i-\frac12}- \frac{a_{i,i+1}}{c_{i+\frac12, i+1}}f_{i+\frac12}
- b_{i,i-\frac12} p_{i-\frac12}   - b_{i,i+\frac12} p_{i+\frac12}
\right)  
\\+& \frac{c_{i+\frac12, i+1}}{x_{i+1}} \left( 
g_{i+1} - \frac{a_{i+1,i}}{c_{i+\frac12, i}}f_{i+\frac12}- \frac{a_{i+1,i+1+1}}{c_{i+\frac32, i+1+1}}f_{i+\frac32}
- b_{i+1,i+\frac12} p_{i+\frac12}   - b_{i+1,i+\frac32} p_{i+\frac32}
\right) =& f_{i+\frac12}
\end{align*}
%

\begin{equation}\label{eq:}
%
\left\{
\begin{aligned}
\end{aligned}
\right.
%
\end{equation}
%


%==========================================
\end{document}  
%==========================================
